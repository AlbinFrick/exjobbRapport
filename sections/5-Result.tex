\section{Result}
% \begin{itemize}
%    \item Introduction of the first prototype
%    \item First test iteration 
%       \item evaluation of test. 
%       \item what is changed and what is kept?

%    \item Second test iteration 
%       \item evaluation of test. 
%       \item what is changed and what is kept?
% \end{itemize}

In this section, the results from the project will be presented. These results are in sections of interviews, prototyping, and usability testing.

\subsection{Initial research}
\label{sub:Initial research}

At the start of the project, a meeting was held with a team at Knowit to determine their needs and their requirements on the project. After some discussion, there was an interest in creating an automated generating of code from their \acrshort{ui} design program, Figma. 

With this in mind, research began on what competitors there were in the field of code generation for web applications, and two competitors were most interesting, Webflow and Visly (see section \ref{sub:Competitors}). Some interesting small projects were also found that were used as inspiration for creating the tool. One project, especially creating color variables from Figma made by Karl Rombauts, was used \cite{rombautsKarlRombautsFigmaSCSSGenerator2021}.

How to encapsulate the Figma design elements into code was one of the most severe difficulties.  Different technologies were considered to find the fit for Knowit's requirements.  These technologies were using a JavaScript framework (such as Angular or React), plain \acrshort{html}-\acrshort{css}-JavaScript, or Web \glspl{component} with LitElement. 

One of the requirements was that the tool could be applicable for all types of projects. Using a JavaScript framework would mean that some projects would not support the tool. Therefore that technology was counted out. Using plain \acrshort{html}-\acrshort{css}-JavaScript could work for all projects, but it would be hard to manage since the generated code would be static and hard to encapsulate. Web \glspl{component} was chosen as the technology for encapsulating the Figma design elements into code. The lightweight JavaScript class LitElement was then chosen to make the creation of web components easier. LitElement is run natively in \acrshort{html}-\acrshort{css}-JavaScript, and therefore works with all projects. LitElement is also much easier to manage than just generating plain \acrshort{html}-\acrshort{css}-JavaScript.


\subsection{Semi-Structured interviews}%
\label{sub:Initial interviews}
The interviews were done to get an understanding of how this tool could be structured around the employee's workflow for all competence groups.

From the interviews, there was found that the majority liked the idea and thought that the created prototype could be a helpful tool.  The interviews involved three different competence groups, designers, front-end developers, and back-end developers. The interviewees that were back-end developers were the ones with the most hesitation. There where much uncertainty about whether or not a tool like this could be helpful. The main concerns were the responsiveness of the \glspl{component} and to have a link between Figma and the \glspl{component} directly. As of now, the \glspl{component} outer container is static, which means that the width or height of the display showing the \gls{component} does not affect it, which often becomes a problem.   

The interviewees that were designers and front-end developers were more positive. They thought that the tool was interesting and could potentially help with communication between them and the developers.

There was no objection to the implementation at the time because this way of working were a lot different from what these employees were used to.  

\subsection{Prototype}%
\label{sub:Prototype}
The prototype created in this project can create web components dynamically from a Figma document, create \acrshort{scss} variables, and \glspl{mixin} from Figmas colors and text styles. The components, variables, and \glspl{mixin} are ready to be used as a package out of the box with \acrshort{npm}. This prototype is available on Knowit-experience-norrlands GitHub page and is free to use and modify under the MIT open source license. 

The first step of creating the tool was discussing with Knowit what could be possible to achieve under the 20 week project time. As seen from the literature study, the big problem was how to condense the elements from the code to be used efficiently. 

The tool that was to be built would fetch data from Figmas \acrshort{rest}-\acrshort{api}, interpret the response from the \acrshort{api}, and build LitElement objects from the interpretation.

Figmas \acrshort{api} was examined to understand what was possible to do with it. Figmas website for developers \cite{figmaFigma} was read through and also some initial \acrshort{http}-requests were sent to the \acrshort{api}, using platform Postman \cite{PostmanCollaborationPlatform}. The initial response from the \acrshort{api} was large. This meant that setting the TypeScript types correct for all values in the response would be inefficient. To mitigate this the Visual Studio Code \cite{VisualStudioCode} extension quicktype \cite{ConvertJSONSwift} was used to generate types from the JSON response. 

From the \acrshort{api} response, we could see that most, or at least enough, of the data were the same as styling in \acrshort{css}. This meant that styling elements with \acrshort{css} were possible from the \acrshort{api}.

The information from Figmas \acrshort{api} was interpreted and stored as classes of colors, typographies, and \glspl{component}. The strategy was to generate a string that contained the LitElement. Essentially the program generated code as a string. This string is later written into a new TypeScript file that is then compiled into a JavaScript file that could be run in a browser.


The prototype has been usability tested, see section \ref{sub:Usability Testing}, which showed indications of being arbitrarily usable for all essential functions. 

\subsection{Usability Testing}%
\label{sub:Usability Testing}
When the tool was functional, two usability tests were done to ensure that the tool was usable for more than just the author. The goal is that a developer should be able to use the tool with just the user guide (README) from github.com. 

\subsubsection{Iteration one}%
\label{ssub:Iteration one}
The first iteration of the test gave a list of flaws with the interface.
\begin{itemize}
   \item The importance of the Auto-layout feature in Figma. The line about it was read but not understood that it was a dependency for the program. 
   \item Hard to differentiate between explanations for Figma and the program.
   \item When setting up a new Figma document, all users had a hard time what they were allowed to name their document.
   \item TypeScript must be installed globally to run the TypeScript compiler. This was not mentioned in the documentation. 
      \begin{itemize}
         \item When the generation has been completed, the TypeScript files must be compiled using the \textbf{T}ype \textbf{S}cript \textbf{C}ompiler (tsc)-command. This needs to be done to use as a package but was misunderstood.
      \end{itemize}
   \item When creating a local \acrshort{npm}-package, there was no way for the users to understand what the name for the package was. 
   \item When styling the parent of the component, the target string is the component's name in \gls{camel}, which was not described. The styling of the parent can also be done with the regular \textit{Style} attribute, which was suggested by one of the users.
\end{itemize}

All of the flaws found could be corrected for the next iteration of the test. The questions after each test revealed that the users thought the prototype had a logical flow but that there was a high learning curve. They liked the use of pictures in the user guide and suggested adding more if possible.  

\subsubsection{Iteration two}%
\label{ssub:Iteration two}
The second iteration of the test had five participants, where one was a developer from Knowit, and four were students of Interaction and Design studying at the University of Umeå. In addition to the first iteration, tasks about Figma styles were added. The second iteration of the test went much more smoothly than the first. All users could create and convert components without any intervention from the test operator. The second iteration still found flaws in the user guide and prototype. These flaws can be seen in the list below:

\begin{itemize}
   \item The users found it hard to find the Auto-layout in Figma.
      \begin{itemize}
         \item The importance of Auto-layout was understood during the second iteration.
      \end{itemize}
   \item When creating a Figma access token, users now found where to create the token but not how to do it.
   \item When using color and text styles, the users found it hard how to implement them. They often confused the implementation of the two. 
   \item When styling or changing text on the component, the examples are based on an image at the start of the user guide. The users did not understand that the image at the start was connected to the examples. A suggestion from the user was to duplicate the image and show it again further down.
   \item Most users had a hard time understanding why and when to compile the TypeScript component files. A suggestion from a user was to incorporate the compilation within the main script of the prototype.
   \item Two users thought that the components should be used with \gls{camel} instead of \gls{kebab} when initiating the component in the \acrshort{html}.
   \item Some users found it hard to find different headings in the user guide. Their suggestion was to make the headings bigger and thought it unnecessary to have them nested.
\end{itemize}

The flaws that were found in the second iteration of the tests could all be fixed. Changing the main script was one of the most significant changes but allowed the user guide to be shortened and the prototype to be run faster.


\subsection{A/B Testing}%
\label{sub:A/B Testing}

SKRIV OM DETTA SÅ DET INTE BLI I "JAG"-FORM OCH INTE LIKA URSÄKTANDE.

Since this was the first time I did a project at this size. I had difficulties to estimating the time of every given task. I did not anticipate the time needed to create a working application to do the tests on. I was really exited at the beginning of the project to produce tangible results that would impact Knowit's workflow in a positive direction. Since there was no time to finish a testable prototype the A/B testing could not be performed during this projects timespan.
