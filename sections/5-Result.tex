\section{Result}
\begin{itemize}
   \item Introduction of the first prototype
   \item First test iteration 
      \item evaluation of test. 
      \item what is changed and what is kept?

   \item Second test iteration 
      \item evaluation of test. 
      \item what is changed and what is kept?
\end{itemize}

For the first period of time a prototype was made. Because of the uncertainty of how the system should be built a experimental model of programming was used. Where the discovery of the how the different parts of the system works was done continuously throughout the implementation. This meant that there aren't any results to show for the first part of the project. 
(Kan nog ta bort detta. Ville bara få ut det.)

\subsection{Initial interviews}%
\label{sub:Initial interviews}
The interviews where done get a sense of how this tool could be structured around the employees work flow.

From the interviews there was found that majority liked the idea and thought that it could be a useful tool. The interviewees that worked with back-end where the one's with most hesitation. There where a lot of uncertainty whether or not a tool like this could be useful. The main concerns were the responsiveness of the components and to have a link between Figma and the components directly.

From the other side of the competence spectra there more a more positive approach. These are the designers of the almost daily work with Figma. They thought that the tool was interesting and that it could potentially help with communication between them and the developers.


\subsection{User Testing}%
\label{sub:User Testing}
When the tool was functional two(three) usability tests where done to ensure that the tool was usable for more than just the author. The thought is that a developer should be able to use the tool with just the userguide (README) from github.com.

\subsubsection{Iteration one}%
\label{ssub:Iteration one}
The first iteration of the test gave a list of flaws with the interface.
\begin{itemize}
   \item The importance of the Auto-layout feature in Figma. The line about it was read but not understood that it was a dependency for the program. 
   \item Hard to differentiate between explanations for Figma and for the program.
   \item When setting up a new Figma document all users had a hard time to what they were allowed to names their document to.
   \item TypeScript must be installed globally to run the TypeScript compiler. This was not mentioned in the documentation. 
   \item sp
\end{itemize}


\begin{itemize}
   \item 
   \item Visa tydligare att Auto-Layout är ett dependency. Visa vad den gör också.
   \item Visa tydligt var gränsen går mellan "Figma förklaringar" och programmet.
   \item Visa att man får välja namnet på sitt dokument.
   \item Skriv att TypeScript måste vara installat globalt.
   \item Förklara tydligare hur tsc fungerar och varför det behövs.
   \item Vid NPM förklara tydligare hur man lägger upp/länkar packet. (package-name)
   \item Styling för föräldren kan skrivas med "style" eller "compName" (camelcase istället för dashed)
\end{itemize}

\subsubsection{Iteration two}%
\label{ssub:Iteration two}




