\section{Introduction}
% In the early 90's when the internet was starting to take shape there were not much thought behind the design of websites. The focus was to distribute information. As time has past and the Internet has grown to be apart of almost all industries and is a big part of a lot of peoples lives, design starts to matter. Today designing and building websites is a  \$??? industry.  

% With the rise of the agile project work form \cite{cohen2004introduction}, where the product is being developed iteratively, the communication between developers and designers must be on point for the project to move fast and smooth.



When building applications for companies there's a lot of moving parts. Most often the project is worked with an agile work form \cite{cohen2004introduction}. This means that the product is evolving, with a tight connection with the customer. Before development starts some research needs to be done to find what the customer wants, how the application should be implemented, and how it should look. When working on a big project the design and the implementation of the product are done by different competence groups, mainly designers, and developers. 

The communication (and or relationship?) between these two parts are essential to creating a good user experience (UX). The communication problem between designers and developers can be compared to the common \textit{impedance mismatch problem} in the data storage realm. Designers and developers are trying to achieve the same thing but see information differently. The front-end developer is trying to figure out how the design should be written in code. How the design elements should or shouldn't effect each other, how each of the should be positioned, etc. The designers see the relationship between the elements, how the dynamic of contrast, colors, and white space is set for the whole application. 

In later years most development are done around components. Whereas components are design and/or functional elements in the application. A component could be a button, a navigation bar, a card\cite{babichSimpleDesignTips2020}, etc. This is done to save code meaning that all components can be reused through out an UI and between UI's. Many companies create big libraries of these components to be used within different projects. This works great for when the design languages. However when a consulting firms clients often does not have the same design language. This results in that the consulting firms often needs to redesign there components between projects. 



% The front-end developer sees how the design elements relate to each other, how they can be written in code and they could be positioned.


\subsection{The Company \& the Problem}
Knowit is an IT company with about 2600 employees with facilities in Sweden, Norway, Denmark, Finland, and Germany. Knowit has three main branches, Experience, Insight, and Solutions. These branches tackle different types of problems, enforce user experience with the customers brand (Experience), create system solutions to help customers digitalize (Solutions), and management-consulting (Insight). This project was done under the Experience branch to help them with an important part of initiating projects.

This branch is consulting firm for customers that need a digital tool with high availability, accessibility, and great user experience. Every project under Knowit experience is run by teams of designers and developers. Where the designers make the design for the applications, including general designing guidelines, for each project. Then the developers implement the design to a functioning application.  

The problem that has araised is that the design and setup for every project is very similar project to project. This is setting up a color palette, typography and basic components such as buttons, forms, etc. When the designer has done this the developer needs to convert this to code. This process is often very similar for every project but needs to be redone because most project doesn't use the same frameworks and/or tools. 

\subsubsection{Knowit Initial Requirements}%
\label{ssub:Knowit Initial Requirements}
The initial requirements from Knowit was to make a design system\cite{fanguyComprehensiveGuideDesign} to solve to problem at hand. These following are those requirements: 
\begin{itemize}
  \item The system needs to be applicable to all types of projects.
  \item The system needs to be modular, you can choose to just use parts of it.
  \item It has to be easy to change global parameters such as colors, fonts, margins, etc.
  \item The system will be open source and easy to make changes to.
  \item The system needs a thorough documentation.
\end{itemize}

After some research the conclusion was made that there would not be enough time to administrate the design system after it was built. Therefor the angle of attack was changed to focus on the making of components. The before mentioned requirements still stood but instead of building a design system we wanted to build an automated process to generate code from a design. 






% \subsection{Aim}%
% \label{sub:Aim}
% The aim of this thesis is to investigate the current practices form design to code and from that investigation create a tool to automate the whole process . 


\subsection{Objective}
\label{sub:Objective}
Most applications is built by components such as buttons, forms, cards\cite{babichSimpleDesignTips2020}, etc. These components are often redesigned and rebuilt from project to project making it bothersome but necessary work.

% The problem that this thesis is trying to solve is to make

The aim of this study is to make this setup time extensively more efficient with a tool that generates components. 

To make this possible the following research questions must be answered:
\begin{itemize}
  \item Is it possible to automate the whole process from UI design program to browser runnable code? 
 \item Can components be built that works for all JavaScript frameworks and static pages?
  \item How can a user-friendly tool be built that automates component generation?  
  \item Will automation between design and development increase communication between designers and developers?
  \item Does this tool speed up the development process and if so how much?
  % \item Could this tool make administration more efficient?
\end{itemize}


\subsection{Demarcations}%
\label{sub:Demarcations}
This automation could be used to create a whole component library and be a part of a whole design system. This project was done under the limited time of 20 weeks. Therefore only a basic component generator will be produced. With \textit{basic} components the only things that will be taken in to account is shape, color, typography and layout. More advanced styling such as gradients, SVGs, etc. will not be apart of the prototype. The prototype will not be full accessibility becuase  the elements used to build the components where only divs, for containers, and/or p-tags, for texts.
The focus was on the data flow from design tool to usable code and therefore the component itself were not required to be perfect. 

