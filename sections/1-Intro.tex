\section{Introduction}
% In the early 90's when the internet was starting to take shape there were not much thought behind the design of websites. The focus was to distribute information. As time has past and the Internet has grown to be apart of almost all industries and is a big part of a lot of peoples lives, design starts to matter. Today designing and building websites is a  \$??? industry.  

% With the rise of the agile project work form \cite{cohen2004introduction}, where the product is being developed iteratively, the communication between developers and designers must be on point for the project to move fast and smooth.


When building applications for companies, the workflow can become complex and hard to manage. The projects are often carried out with an agile work form \cite{cohen2004introduction}. Agile work form means that the product evolves, with a tight connection with the customer. Before development starts, some research needs to be done to find what the customer wants, how the application should be implemented, and how it should appear. When working on a big project in general, the design and the implementation of the product are done by different competence groups, mainly designers and developers. 

Communication between designers and developers is essential to constructing a good user experience (UX). The communication problem between designers and developers can be compared to the typical \textit{impedance mismatch problem} in the data storage realm \cite{irelandClassificationObjectRelationalImpedance2009}. Designers and developers are trying to achieve the same thing but see information differently. The front-end developer is trying to figure out how the design should be written in code. How the design elements should or should not affect each other, how each of them should be positioned, and so forth. The designers see the relationship between the elements, how the dynamic of contrast, colors, and white space is set for the whole application. 

In the early 1990's the internet was starting to take shape \cite{WebD2BriefHistory}. At this time, distributing information was the main focus of the internet. The way to do this was by using \acrfull{html}, which was released in 1993 \cite{WebD2BriefHistory}. This is a static markup language that allows the user to display data to a browser. This data is formatted and displayed as a document not much different from this report. To make more complex "things", such as manipulating the DOM dynamically during runtime, with the web a scripting language was created called LiveScript, later renamed to JavaScript \cite{JavaScript2021}. To make the web more approachable and easier to use a styling language for \acrshort{html} was created by Håkion Wium Lie and Bert Bos in 1995 and released in 1996 \cite{BriefHistoryCSS}. Nowadays, 97\% of the \acrlong{www} is built by these three languages \cite{JavaScript2021}.

A lot has happened in the last 30 years, where most development for web applications is modular and built up by \glspl{component}. \Glspl{component} are design and functional elements in the application. For example, a \gls{component} could be a button, a navigation bar, a card \cite{babichSimpleDesignTips2020}, etc. \Glspl{component} are created to save code, meaning that all \glspl{component} can be reused throughout a user interface (\acrshort{ui}) and between \acrshort{ui}'s. Many companies create extensive libraries of these \glspl{component} to be used within different projects. Using \gls{component} libraries works well when the design language is the same for all projects. However, when a firm has many different clients, the projects can differ in their design language. The differences in design result in the consulting firms often needing to redesign their \glspl{component} between projects. 



% The front-end developer sees how the design elements relate to each other, how they can be written in code and they could be positioned.


\subsection{The Company \& the Problem}
\label{sub:company}
% Knowit is an IT company with about 2600 employees in Sweden, Norway, Denmark, Finland, and Germany. Knowit has three main branches, Experience, Insight, and Solutions. These branches handle different types of problems, enforce user experience with the customers brand (Experience), create system solutions to help customers digitalize (Solutions), and management-consulting (Insight). This project was carried out under the Experience branch as an important building block for being able to find a more effective way to initiate new web building projects.

% This branch is a consulting firm for customers that need digital tools with high availability, accessibility, and good user experience. Every project under Knowit experience is run by teams including designers and developers. 

% Where the designers make the design for the applications, including general designing guidelines, for each project. Then the developers implement the design to a functioning application.  

% The problem that has arisen is that the design and setup are very similar for each project. This is setting up a color palette, typography, and basic \gls{component} such as buttons, forms, etc. When the designer has done this the developer needs to convert this to code. This process is often very similar for every project but needs to be redone because most projects do not use the same frameworks and/or tools. 

Knowit is an IT company with about 2600 employees in Sweden, Norway, Den- mark, Finland, and Germany. Knowit has three main branches, Experience, Insight, and Solutions. These branches handle different types of problems, enforce user experience with the customers brand (Experience), create system solutions to help customers digitalize (Solutions), and management-consulting (Insight). This project was carried out under the Experience branch as an important building block to find a more effective way to initiate new website building projects. 

This branch is a consulting firm for customers that need digital tools with high availability, accessibility, and good user experience. Teams, including designers and developers, run every project under the Knowit experience branch. The designers make the design for the applications, including general designing guidelines, for each project, then the developers implement the design to a functioning application. 

Knowit uses an agile work form derived from the manifesto for Agile Software Development \cite{ManifestoAgileSoftware}. This is an iterative work form where the product is continuously improved with tests, analysis, and customer input. A simplified visualization of this way to work is shown in figure \ref{fig:agile}.

\begin{figure}[H]
  \centering
  \includegraphics[width=0.8\linewidth]{images/agile.png}
  \caption{Simplified illustration of the agile workflow for Knowit.}%
  \label{fig:agile}
\end{figure}

 The setup consists of setting up a color palette, typography, and fundamental \glspl{component} such as buttons, forms, etc. When the designer has done this, the developer needs to convert this to code. These components are often built with the same fundamental elements. Hence the process is often similar for each project but needs to be redone because most projects do not use the same frameworks and tools. The problem that has arisen is that the setup is similar for each project. This thesis is therefore focused on the handover from design to development, the most right arrow in figure \ref{fig:agile}.

\subsubsection{Knowit Initial Requirements}%
\label{ssub:Knowit Initial Requirements}
The initial idea from Knowit was to make a design system \cite{fanguyComprehensiveGuideDesign} that followed the following requirements:
\begin{itemize}
  \item The system needs to be applicable for all types of projects.
  \item The system needs to be modular. The choice to use parts of it must be an option.
  \item It has to be easy to change global parameters such as colors, fonts, margins, etc.
  \item The system must be open source.
  \item The system needs thorough documentation.
\end{itemize}

After some research, we concluded that there would not be enough time to administrate the design system after this project. Hence the focus of the thesis was changed from making a design system to streamlining the creation of \glspl{component}. This change was significant, but the before-mentioned requirements still stood for the new focus. 

% \subsection{Aim}%
% \label{sub:Aim}
% The aim of this thesis is to investigate the current practices form design to code and from that investigation create a tool to automate the whole process . 



\subsection{Objective}
\label{sub:Objective}
Most applications are built by \glspl{component} such as buttons, forms, cards \cite{babichSimpleDesignTips2020}, etc. These \glspl{component} often need to be redesigned and rebuilt from project to project, making it bothersome but necessary work. 

This study aims to make this setup time extensively more efficient with a tool that generates \glspl{component}. 

To reach the aim of the study, we need to answer the five questions stated below:  

\begin{itemize}
  \item Is it possible to automate the whole process from UI design program to browser runnable code? 
  \item Can \glspl{component} be built that works for all JavaScript frameworks and static pages? 
  \item How can a user-friendly tool be built that automates component generation? 
  \item Will automation between design and development increase communication between designers and developers? 
  \item Does this tool make the development process more effective?
\end{itemize}

\subsection{Demarcations}%
\label{sub:Demarcations}
This automation could be used to create a whole component library and be a part of a whole design system. However, this project was done within a limited time of 20 weeks. Therefore only a fundamental component generator will be produced. With these fundamental \glspl{component}, the only things that need to be considered are shape, color, typography, and layout. More advanced styling, such as gradients, \acrshortpl{svg}, etc., will not be a part of the prototype. The prototype will not have full accessibility because the \acrshort{html} elements used to build the \glspl{component} will be \glspl{div-tag} for containers and \glspl{p-tag} for texts. The focus for the prototype was on how the data flows from the design software to usable code.




