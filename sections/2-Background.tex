\section{Background}
\textbf{\textit{present contextual or prerequisite information that is important or essential to understand the main body of the thesis}}

% The process of creating a web-application is done in three major steps: 
% \begin{itemize}
%   \item Design
%   \item Develop
%   \item Publish
% \end{itemize}

% To create a tool that can automate the process of making design components to functional code these main areas must be considered. 



% \textit{Don't know were to put this}
% \subsection{Relationship Between Designer and Developer}%
% \label{sub:Relationship Between Designer and Developer}
% To create a great product the collaboration between designers and developers is essential.

% \textit{"Without the proper alignment and synergy of skillsets, the user experience you provide your users will lack efficiency, engagement, and value."} - Khamdamova, \cite{RelationshipDesignersDevelopers} 

% \subsection{Design}%
% \label{sub:Design}
% A fairly regular way of making the design of websites is to first create prototypes with very low details and then build in details from there. This is LoFi to HiFi prototyping. Where the customer can be a part of the steps to create something that they want and can be done. The medium with which these prototypes are created varies but most often than not a UI design application is used to make the final HiFi-prototype. 


% \subsubsection{UI Design Applications}%
% \label{ssub:Apps}

% Assesemnt of Figma vs Sketch \cite{SketchVsFigma0200} 


% Explanation of how these work and why they are used.


% \subsubsection{Figma}%
% \label{sub:Figma}
% Figma is a UI Design application which is web based. Meaning that the whole program is run over network. 

% Figma has an open REST-API, (Representational State Transfer), that supply's the information of the Figma document over the Internet \cite{figmaFigma, RepresentationalStateTransfer2021}. Figma is also web-based meaning that the program runs over a network connection. Because of this the API is constantly updated after each change in the design, which great for collaboration and getting the API constantly updated.

% \subsubsection{Webflow}
% Webflows website: \cite{ResponsiveWebDesign} 

% \subsubsection{Visly}%
% \label{ssub:Visly}
% Visly website: \cite{vislyVisly} 


% \subsubsection{Components}%
% \label{sub:Components}

% \subsection{Develop}%
% \label{sub:Develop}


% \subsubsection{REST API}%
% \label{sub:REST API}
% \cite{RepresentationalStateTransfer2021} 


% \subsubsection{Styling}%
% \label{sub:styling}
% Styling meaning how the design is implemented to a real application.

% \subsubsection{CSS Methodologies}%
% \label{ssub:methodologies}

% Explanation for BEM\cite{contributorsBEMBlockElement} 

% % \subsubsection{SCSS}%
% % \label{ssub:SCSS}
% % \cite{SassSyntacticallyAwesome} 

% % \subsubsection{Webpack}%
% % \label{sub:Webpack}



% \subsubsection{Web Components}%
% \label{sub:Web Components}
% Web components are custom made components that are reusable.

% Web components are built in to HTML5 which makes them work natively in every browser and in congestion with all frameworks.

% This \cite{WebComponentsMDN} could be very useful.
%   \paragraph{\ref{sub:Web Components}.1 Custom Elements}%
%   \label{ssub:Custom Elements}

%   \paragraph{\ref{sub:Web Components}.2 Shadow DOM}%
%   \label{ssub:Shadow DOM}

%   \paragraph{\ref{sub:Web Components}.3 HTML Templates}%
%   \label{ssub:HMTL Templates}

% \subsubsection{LitElements}%
% \label{ssub:LitElements}
% This is a great light weight class to simplify making web components.
% \cite{polymerLitElement} 



% \subsection{Publish}%
% \label{sub:Publish}
% To be able to use the components created from the designer the developers need to be able to get a hold of them. A lot of reusable code that is crated for the web can be accessed from the internet with the help of package managers.

% \subsubsection{Package Manager}%
% \label{sub:Package Manager}
% \cite{PackageManager2020} 

% A package manager is a way to install and update programs with ease.
