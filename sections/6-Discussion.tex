\section{Discussion}
% Discussion about the results and why it turned out the way it did. Hopefully this created a great tool that will take over the world

In this section a discussion about the results of the study will be taken place. Furthermore a discussion on what could have been done differently. What was done well and what could have been improved for all the stages of the study. The prototype that was produced, the usability, and the A/B testing that unfortunatly could not be executed.

\todo{Min prototyp är inte låst till ett visst ramverk.}
\todo{Pandemi (interviews och tester digitalt)}

\subsection{The Prototype}%
\label{sub:The Prototype}
The prototype that was produced during this study has a 

\subsection{Interviews}%
\label{sub:Interviews}

The intention of the interviews was to find out how the employees of Knowit work, what their day-to-day routines were. The hope was that the interviews could help shape the prototype into something that felt more familiar to them. There was not as much gathered from the interviews as hoped. Because of the new way of making the components with this tool was very new it was hard to get any similarities with the work they do today. The way of using components and controlling them through properties is something they are working with in some react projects. This was already planned to be done for the components in the prototype but was a good verification.

The interviews was certainly interesting. There was a shown that a tool with this overlap could help with communication shown from most competence groups.

The assumption for the interviews was the people working with back-end would be most exited because with the tool they would not have to make any adjustments from the front-end. When using the tool there are more things that the designer has to thing about. How the structure of their component are in Figma not just visually. Also the need of Auto-layout that needs to be used for the prototype to work. Because of this another assumption for the interviews was the designers would be more negative to using this tool. 

Both of these assumptions was wrong and from the interviews done the opposite was shown. To be get a statistical significance of this would require more data points, e.g. more interviews from each competence area. This shows that the tool could be used to increase communication between competence areas. Which could result in faster development and a better working environment. More of this in section \ref{sub:Future Work}.

\subsection{Usability testing}%
\label{sub:Usability testing}
Wow the usability testing it sure was fun and frustrating.


\subsection{A/B testing}%
\label{sub:A/B testing}

\section{Conclusion}
\label{sub:conclusion}

\section{Future Work}%
\label{sub:Future Work}
Futher developement should be done with:
\begin{itemize}
   \item Gradients
   \item package.json for output
   \item separat kunna få ut variabler 
\end{itemize}


