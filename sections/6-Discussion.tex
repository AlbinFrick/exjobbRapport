\section{Discussion}
% Discussion about the results and why it turned out the way it did. Hopefully this created a great tool that will take over the world

In this section a discussion about the results of the study will be taken place. Furthermore a discussion on what could have been done differently. What was done well and what could have been improved for all the stages of the study. The prototype that was produced, the usability, and the A/B testing that unfortunatly could not be executed.


\subsection{The Prototype}%
\label{sub:The Prototype}
The most time and effort for this project went in to creating the prototype. 
There was a lot of problems that needed to be solved where the biggest one was; How can a component be built that works for all JavaScript frameworks including static pages. To figure this out the research went in to find something that works for static pages. Because if something works natively within HTML-CSS-JavaScript that has a much higher chance of working together with frameworks. The web component was a clear choice.

Web component are as explained in the theory a mixture of three technologies, custom elements, shadow DOM, and HTML templates. This would work natively for all frameworks and static pages. The problem with doing it this way is that there is complicated to make these components loosely couple between projects. The LitElement class from Polymer was the solution. This also made the creation of the components much easier. A problem with LitElement was that to use them you need to handle \textit{open imports}. This is when the import does not target a file or a function from a module. This is only supported when running in node.js not in browsers at the moment. To combat that a bundler must be used, such as Webpack or Rollup.This was not optimal but the upside of LitElement still trumped this downside. This answers two of the research questions stated at the beginning of the project: \textit{Can components be built that works for all JavaScript frameworks and static pages?} and \textit{Is it possible to automate the whole process from UI design program to browser runnable code?}.


One of the biggest struggles with creating the tool was to make sure that the prototype built the right amount of elements in the right place. This was solved with recursive functions that ''digs'' down to the ''youngest'' child and builds the elements from there an goes from generation to generation. This is a good way of building the components but it can become a performance taxing function if the components becomes to big. This has not been tested but there has been indications that there needs to be done optimization of these functions to make the prototype viable for bigger components and projects.

The prototype has some features that unfortunately was not tested. Such as slotting. Where the user can decide a slot in the component where you can insert en element or another component into it. This feature was thought to be necessary to do the required task in later tests which was wrong. The time spent on developing the feature could have gone to testing which would have been more efficient.



\subsection{Interviews}%
\label{sub:Interviews}

The intention of the interviews was to find out how the employees of Knowit work, what their day-to-day routines were. The hope was that the interviews could help shape the prototype into something that felt more familiar to them. There was not as much gathered from the interviews as hoped. Because of the new way of making the components with this tool was very new it was hard to get any similarities with the work they do today. The way of using components and controlling them through properties is something they are working with in some react projects. This was already planned to be done for the components in the prototype but was a good verification.

The interviews was certainly interesting. There was a shown that a tool with this overlap could help with communication shown from most competence groups.

The assumption for the interviews was the people working with back-end would be most exited because with the tool they would not have to make any adjustments from the front-end. When using the tool there are more things that the designer has to think about. How the structure of their component are in Figma not just visually. Also the need of Auto-layout that needs to be used for the prototype to work. Because of this another assumption for the interviews was the designers would be more negative to using this tool. 

Both of these assumptions was wrong and from the interviews done the opposite was shown. To be get a statistical significance of this would require more data points, e.g. more interviews from each competence area. This shows that the tool could be used to increase communication between competence areas. Which could result in faster development and a better working environment. More of this in section \ref{sub:Future Work}.

\subsection{Usability testing}%
\label{sub:Usability testing}
The usability testing in this thesis was done with the more industrial style of working with the resources at hand. The project had 20 weeks and was done during the Covid 19 pandemic meaning that all testing was done remotely. This also meant that it was harder to get users to do the tests. All test that was made where a walk through for the whole prototype, from design to web component. It was considered to make smaller tests to check individual parts of the system. This was not used because there wasn't enough users and time. Because of the broadness of the prototype it was also highly important that the users could make it from start to finish without any help. 

From the two iterations we can see that there were seven flaws found each iteration. At first glance this seems as though the fixes done after the first iteration did not work. The test was altered with added tasks in the second iterations which means that the average flaws per task went down from the first to second iteration. 

To say that the whole system is user friendly would be an over statement. What the usability tests has shown is that for the basic functions such as setting up a Figma document and converting its components. The tests also shows that altering the components after they have been generated is logical for the users. 

I would argue that this is a very efficient way of building a tool that automates component generation. Thereby answering one of the research questions: \textit{How can a user-friendly tool be built that automates component generation?}

\subsection{A/B testing}%
\label{sub:A/B testing}


\begin{itemize}
 \item Will automation between design and development increase communication between designers and developers?
  \item Does this tool speed up the development process and if so how much?
\end{itemize}

\section{Conclusion}
\label{sub:conclusion}

\section{Future Work}%
\label{sub:Future Work}
Futher developement should be done with:

The next step in the development would be to create a graphical user interface. This would be a lot easier to steer the user in the right direction and making the user experience a lot better. The learning curve for the user could be shortened to be most about using and altering the generated components. 

\begin{itemize}
   \item Gradients - effects - rotation...
   \item package.json for output
   \item separat kunna få ut variabler 
   \item Graphical interface 
   \item More general approach, not using Auto-layout
   \item 
\end{itemize}


