\section{Method}

\textit{
This system was thought of as a tool to be used in many different projects. Because of this the mark on accessibility was big. The developer should not be constrained on what framework or lack there of he or she is using. This should work on all platforms.
}


\subsection{Workflow}%
\label{sub:workflow}

\begin{enumerate}
  \item Literature study / External analysis
    \begin{itemize}
      \item Research which tools to use
    \end{itemize}
  \item Preform interviews/collect data for the workflow that is used today in the company.
  \item Create a tool that uses Figmas API to create components (testing accessibility/"code quality" trough out)
  \item Preform user testing to ensure usability. 
  \item Finalize tool with certain demarcations.
  \item Preform AB testing to study efficiency with and without the new tool for starting up a project.
\end{enumerate}

\subsection{Literature study}%
\label{sub:Literature study}
How to communicate? 
How to build relationship? 
How to easier build products?

The subject at hand was thoroughly 


\subsection{Creating the tool}%
\label{sub:}
% After the interviews and research phase a clear understanding of the tools that would be used was had. (<--haha what) For creating the tool it self TypeScript was used to be able to use types to find problems before they crashed the code. LitElement was then used to 


Meetings with Knowit and where discussions of what could be possible to do under the 30 weeks of work that should be carried out. As seen from the literature study the big problem was how to condense the elements in from the code to easly be used. 

Identifying te tools that should be used.
\begin{itemize}
  \item TypeScript - for building the tool.
  \item Exploring Figmas API.
  \item LitElement - For building the components. 
  \item Building the html elements and styling with recursive functions
\end{itemize}

To build the tool an experimental approach with \textbf{TypeScript} was used. This means that code where tested and possibilities where explored during the building of the program. Why \textbf{TypeScript}? because of the unknown possibilities at the start of the development there where some thoughts on making the generator part of the component itself. Thereby the browser had to be able to run the code which made JavaScript a good match. JavaScript is a programming language that reports its error much later than many other languages. Variables can take any shape in JavaScript. That is for instance a number or a string. This can at first seem as something good that the language is highly dynamic but this is also very error prone. Where most of the errors are discovered after the program is run. 


\textit{Incremental testing is required. Due to its highly permissive, error-tolerant nature, JavaScript programming requires an incremental, evolutionary approach to testing as well. Since errors are reported much later than usual, by the time an error is reported it is often surprisingly difficult to pinpoint the original location of the error. Error detection is made harder by the dynamic nature of JavaScript, for instance, by the option to change some of the system features on the fly. Furthermore, in the absence of strong, static typing, it is possible to execute a program and only at runtime realize that some parts of the program are missing. For all these reasons, the best way to write JavaScript programs is to proceed step by step, by writing (and immediately testing) each new piece of code. If such an incremental, evolutionary approach is not used, debugging and testing can become quite tedious even for relatively small JavaScript applications.} \cite{taivalsaari2008web} 





\subsection{Interviews from Knowit}%
\label{sub:}
To get an understanding of what the tool should be able to do and how it should operate. Semi-structured interviews were held with (??seven??) employees of Knowit.

\subsection{User testing}%
\label{sub:userTesting}

\subsection{ Finalized Prototype }%
\label{sub:finalizingPrototype}

\subsection{ AB Testing }%
\label{sub:mABTesting}
