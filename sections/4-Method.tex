\section{Method}
% \textit{
% This system was thought of as a tool to be used in many different projects. Because of this the mark on accessibility was big. The developer should not be constrained on what framework or lack there of he or she is using. This should work on all platforms.
% }

In this section, the methods used to answer the research questions (see section \ref{sub:Objective}) will be presented. At the beginning of the project, a literature study was performed. This to get an overview of what similar work existed in the field. Knowit, the company connected to the study, wanted an easier way to create components for web development projects. A tool was built to fulfill this want. 

Throughout this process, Knowit was involved with semi-weekly checkups for support. Semi-Structured interviews were carried out on the employees of Knowit to steer the development of the tool to fit their needs. 

When a prototype was somewhat complete, iterations of usability testing were performed. This was done to ensure that the tool was usable for more people than just the author. Lastly, to investigate what impacted the created tool could have, an A/B test was scheduled to be performed.

\subsection{Initial Research}%
\label{sub:Initial Research}

At the start of the project, a meeting was held with a team at Knowit to determine their needs and their requirements on the project. After some discussion, there was an interest in creating an automated generating of code from their UI design program, Figma. 

With this in mind, research began on what competitors there were in the field of code generation for web applications, and two competitors were most interesting, Webflow and Visly (see section \ref{sub:Competitors}). Some interesting small projects were also found that were used as inspiration for creating the tool. One project, especially creating color variables from Figma made by Karl Rombauts, was used \cite{rombautsKarlRombautsFigmaSCSSGenerator2021}.

How to encapsulate the Figma design elements into code was one of the most severe difficulties.  Different technologies were considered to find the fit for Knowit's requirements.  These technologies were using a JavaScript framework (such as Angular or React), plain HTML-CSS-JavaScript, or Web components with LitElement. 

One of the requirements was that the tool could be applicable for all types of projects. Using a JavaScript framework would mean that some projects would not support the tool. Therefore that technology was counted out. Using plain HTML-CSS-JavaScript could work for all projects, but it would be hard to manage since the generated code would be static and hard to encapsulate. LitElement was chosen as the technology for encapsulating de Figma design elements into code. LitElement is run natively in HTML-CSS-JavaScript, and therefore works with all projects. LitElement is also much easier to manage than just generating plain HTML-CSS-JavaScript.


% \begin{itemize}
%   \item Möte med knowit om behov och krav.
%   \item vad fanns för konkurrenter.
%   \item vilka liknande projekt fanns?
%   \item Vilka verktyg passade in knowits behov.

% \end{itemize}


\subsection{Creating the tool}%
\label{sub:}
% After the interviews and research phase a clear understanding of the tools that would be used was had. (<--haha what) For creating the tool it self TypeScript was used to be able to use types to find problems before they crashed the code. LitElement was then used to 


The first step of creating the tool was discussing with Knowit what could be possible to achieve under the 20 week project time. As seen from the literature study, the big problem was how to condense the elements from the code to be used efficiently. 

The tool that was to be built would fetch data from Figmas REST-API, interpret the response from the API, and build LitElement objects from the interpretation.

The tool was built using an experimental approach, meaning that code was tested, and possibilities were explored during the development of the program. TypeScript was chosen as the programming language because Knowit uses it in most projects and would therefore be familiar to them.

Figmas API was examined to understand what was possible to do with it. Figmas website for developers\cite{figmaFigma} was read through and also some initial HTTP-requests were sent to the API, using platform Postman \cite{PostmanCollaborationPlatform}. The initial response from the API was quite large. This meant that setting the TypeScript types correct for all values in the response would take much time. To mitigate this the Visual Studio Code \cite{VisualStudioCode} extension quicktype\cite{ConvertJSONSwift} was used to generate types from the JSON response. 

From the API response, we could see that most, or at least enough, of the data were the same as styling in CSS. This meant that styling elements with CSS were possible from the API.

The information from Figmas API was interpreted and stored as classes of colors, typographies, and components. The strategy was to generate a string that contained the LitElement. Essentially the program generated code as a string. This string is later written into a new TypeScript file that is then compiled into a JavaScript file that could be run in a browser.

\subsubsection{Building the HTML for the component}%
\label{ssub:building the skeleton of the component}
The Figma components have a parent-children structure, meaning that all elements in the component are related. A recursive function was used to search through each element and their children dynamically. This replicates the relationship between the elements and thereby creating a proper HTML structure.


\subsubsection{Styling the Component}%
\label{ssub:Styling the component}
One of the requirements from Knowit was that the components should be easy to alter. This meant that the style of the developer must be able to alter the style of the generated component. 

In the first attempt to make this happen, all CSS rules for an element were assigned a \textit{property} in LitElement. A property allows the developer to pass data into the LitElement. By assigning a property for all CSS rules of the element, the style could be changed after generating the component.

This was later redesigned because the user could not add \textit{new} CSS rules to the component if they wished to. This problem was fixed by storing the CSS rules in maps\cite{ArrayPrototypeMap} and pushing them into the correct CSS selector. Instead of creating a property for each style attribute, only one property for each element is created. If the user wishes to add or change the styling of a component, they target the Figma element as an attribute to the component and inserts regular CSS. The component then creates a duplicate of the styling map for the targeted element and inserts the new styling attributes into the component.


\subsubsection{Variables}%
\label{ssub:Variables}
Figma has a feature called styles. This is a way for the user to store and reuse colors, texts, and effects. This is something that is also very normal to do in a developer environment. Therefore a decision was made to create an SCSS \textit{variable} file where these would be stored. Because of the time constraint of the project, only colors and texts were implemented. This was done similarly to the components where the ''code'' for the SCSS variables was written to a string that later was written into an SCSS file. 


\subsubsection{Open Source}%
\label{ssub:Open Source}
One of Knowit's initial requirements where that the software produced should be open source (see section \ref{ssub:Open Source}). The project was handed access to Knowit Experience Norrlands GitHub page. The software was uploaded publicly to this GitHub repository. The MIT\cite{MITLicenseOpen} open source license where then attached to the repository stating that the software is free to use but has no liability or warranty.
 
\subsubsection{Userguide}%
\label{ssub:Userguide}
The program built does not have a graphical user interface, again because of the time constraint. The user is instead using a command-line interface (CLI). This makes it a bit harder to learn because there are no visual queues of what to input into the program.  A user guide was created in the form of a README on GitHub\cite{BuildSoftwareBetter} to solve this issue. Alon side the prototype itself, this user guide was altered after the usability tests.

% knowit uses a supercharged version of css called scss in most of their projects (section \ref{sub:sass}). scss handles variables in a more intuitive way than regular  css which made it the clear choice for storing styling variables moving forward.


\subsection{Semi-Structured Interviews}%
\label{sub:inteviews}
This tool is intended to be used by people in many different areas of expertise, from designers to front-end developers to back-end developers. A semi-structured interview model was used \cite{galletta2013mastering} to understand the work these people do in their respective roles to be able to build/develop a tool that is functional for all parts of the workflow (all the different roles in the workflow). The semi-structured interview was carried out with a script of questions that are asked to every interviewee. Unlike the structured interview, the semi-structured interview allows for further explanation and follow-up questions from the interviewer. These interviews were done with seven employees of Knowit Experience Umeå and Sundsvall. 

Because of the broad nature of the tool created, it was important to get participants that worked with all affected areas of expertise. The interviews were done digitally over Microsoft Teams. The script used for these interviews can be found in the appendix (The script is only in Swedish). (\todo{lägg till appendix}).


% \subsection{interviews from knowit}%
% \label{sub:}
% to get an understanding of what the tool should be able to do and how it should operate. semi-structured interviews were held with (??seven??) employees of knowit.





\subsection{Usability Testing}%
\label{sub:usertesting}
Usability tests were done to make sure that the prototype was usable for somebody else than the author. Furthermore, to set up the prototype for further testing regarding the effectiveness of the prototype.

Two iterations of usability testing were carried out on nine participants, four in the first iteration and five in the second iteration. The participants had to have a background in web development, NPM, and Figma. Therefore, the participants chosen for the tests were employees of Knowit and students from the interaction and design program at the University of Umeå. The test was designed as a scenario with four different tasks. The participant first got a link to the GitHub repository where the tool and the user guide were situated. The tasks were to create a viable Figma component that could be converted to a web component using the created tool. After that, the participant should install the tool on their computer. Use the tool to convert the Figma component and then insert the component, using NPM locally, in a test project supplied by the test administrator.

This was a way to test the whole chain from designing Figma components, converting these to web components, and to finally using the components in a project. These tests could also be used to determine what was working and not in the user guide. After the tasks were done, the questions about the experience were asked. Finally, the test administrator opened up for suggestions regarding improvements to the tool or the user guide. 

The test script was altered after the first iteration to target more features of the prototype. These features were the use of color and text styles. The questions from the first iteration were still asked. This was done to confirm that the changes made between iteration one and iteration two were effective. 







\subsection{ A/B Testing }%
\label{sub:ab-testing}

To have a metric that can be measured, discussed, and statistically verifiable whether or not the prototype is effective in real-world use, A/B tests will be performed.  The primary goal of the test is the efficiency of the prototype. This will be an abnormal test, where instead of changing a small variable in an interface, the whole system will be tested. 

The task for the participants of the test is to create a simple website from a description that can be found in the appendix. The test participants will be working in pairs of designers and developers, where they ought to collaborate to complete the task. 

The ''control'' variant for the test will be creating a website as the participants are used to, and the ''challenger'' will be creating the website using the prototype. To understand when a test has ''passed'' a 


